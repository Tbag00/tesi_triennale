\chapter{BERTopic}
Descrivere l'architettura di BERTopic e le motivazioni per cui è stata scelta per l'analisi degli annunci.

\section{Pipeline}
Presentare la pipeline end-to-end, evidenziando gli input, le trasformazioni intermedie e l'output finale del modello.

\subsection{Embedding}
Indicare come vengono generati gli embedding con BERT (o varianti), eventuali fine-tuning e impostazioni rilevanti.

\subsection{Dimensionality Reduction}
Illustrare l'algoritmo di riduzione dimensionale adottato (es. UMAP), i parametri principali e l'impatto sulla qualità dei topic.

\subsection{Clustering}
Spiegare il metodo di clustering (es. HDBSCAN), i criteri di scelta dei parametri e come viene determinato il numero di topic.

\subsection{Tokenizer}
Dettagliare il tokenizer utilizzato, le strategie di pre-processing e qualsiasi personalizzazione per il dominio degli annunci di lavoro.

\subsection{Cosa si può migliorare}
Discutere criticità note, possibili ottimizzazioni della pipeline e idee per estensioni future di BERTopic nel progetto.
